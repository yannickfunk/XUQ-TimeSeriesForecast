% This file was created with tikzplotlib v0.10.1.
\begin{tikzpicture}

\definecolor{darkgray176}{RGB}{176,176,176}
\definecolor{green}{RGB}{0,128,0}
\definecolor{steelblue31119180}{RGB}{31,119,180}

\begin{groupplot}[group style={group size=1 by 3}]
\nextgroupplot[
scaled x ticks=manual:{}{\pgfmathparse{#1}},
tick align=outside,
tick pos=left,
title={sine\_noisy - predictions},
x grid style={darkgray176},
xmin=80128.85, xmax=80286.15,
xtick style={color=black},
xticklabels={},
y grid style={darkgray176},
ymin=-6, ymax=6,
ytick style={color=black}
]
\path [draw=green, fill=green, opacity=0.2]
(axis cs:80232,-4.20053672790527)
--(axis cs:80232,-4.23965358734131)
--(axis cs:80233,-4.90621280670166)
--(axis cs:80234,-5.22802448272705)
--(axis cs:80235,-5.04703283309937)
--(axis cs:80236,-4.58175992965698)
--(axis cs:80237,-3.59521579742432)
--(axis cs:80238,-2.29574346542358)
--(axis cs:80239,-1.00480675697327)
--(axis cs:80240,0.159684270620346)
--(axis cs:80241,1.06658411026001)
--(axis cs:80242,1.65422773361206)
--(axis cs:80243,2.1299831867218)
--(axis cs:80244,2.4980103969574)
--(axis cs:80245,2.43843984603882)
--(axis cs:80246,2.23628973960876)
--(axis cs:80247,1.60530924797058)
--(axis cs:80248,0.795435070991516)
--(axis cs:80249,-0.131712213158607)
--(axis cs:80250,-1.30181086063385)
--(axis cs:80251,-2.40672397613525)
--(axis cs:80252,-3.32111358642578)
--(axis cs:80253,-3.97964954376221)
--(axis cs:80254,-4.2949595451355)
--(axis cs:80255,-4.12299156188965)
--(axis cs:80256,-3.74082565307617)
--(axis cs:80257,-3.02554535865784)
--(axis cs:80258,-2.20900535583496)
--(axis cs:80259,-1.24439585208893)
--(axis cs:80260,-0.511167645454407)
--(axis cs:80261,0.122717499732971)
--(axis cs:80262,0.298103272914886)
--(axis cs:80263,0.323616743087769)
--(axis cs:80264,0.148480415344238)
--(axis cs:80265,-0.115378737449646)
--(axis cs:80266,-0.475630640983582)
--(axis cs:80267,-0.715853691101074)
--(axis cs:80268,-0.911928534507751)
--(axis cs:80269,-1.26925992965698)
--(axis cs:80270,-1.39395868778229)
--(axis cs:80271,-1.82951688766479)
--(axis cs:80272,-2.30329704284668)
--(axis cs:80273,-2.62363815307617)
--(axis cs:80274,-2.81754493713379)
--(axis cs:80275,-2.70026588439941)
--(axis cs:80276,-2.37305617332458)
--(axis cs:80277,-1.88719391822815)
--(axis cs:80278,-1.36652612686157)
--(axis cs:80279,-0.656579792499542)
--(axis cs:80279,-0.51450127363205)
--(axis cs:80279,-0.51450127363205)
--(axis cs:80278,-1.35043716430664)
--(axis cs:80277,-1.78554105758667)
--(axis cs:80276,-2.14954209327698)
--(axis cs:80275,-2.41167640686035)
--(axis cs:80274,-2.40260314941406)
--(axis cs:80273,-2.25185680389404)
--(axis cs:80272,-2.00895023345947)
--(axis cs:80271,-1.60335326194763)
--(axis cs:80270,-1.04141390323639)
--(axis cs:80269,0.0446119904518127)
--(axis cs:80268,0.967649817466736)
--(axis cs:80267,1.96669864654541)
--(axis cs:80266,2.61236476898193)
--(axis cs:80265,2.86370658874512)
--(axis cs:80264,2.89764785766602)
--(axis cs:80263,2.47590517997742)
--(axis cs:80262,1.74710297584534)
--(axis cs:80261,0.954625487327576)
--(axis cs:80260,-0.0175766348838806)
--(axis cs:80259,-1.049112200737)
--(axis cs:80258,-2.04559230804443)
--(axis cs:80257,-2.93242859840393)
--(axis cs:80256,-3.64769220352173)
--(axis cs:80255,-4.08717823028564)
--(axis cs:80254,-4.18298482894897)
--(axis cs:80253,-3.82220840454102)
--(axis cs:80252,-3.07010841369629)
--(axis cs:80251,-2.25412178039551)
--(axis cs:80250,-1.16542375087738)
--(axis cs:80249,-0.00866986811161041)
--(axis cs:80248,1.06844198703766)
--(axis cs:80247,2.09449076652527)
--(axis cs:80246,2.98293614387512)
--(axis cs:80245,3.36235380172729)
--(axis cs:80244,3.28461384773254)
--(axis cs:80243,2.99862027168274)
--(axis cs:80242,2.44559717178345)
--(axis cs:80241,1.52339148521423)
--(axis cs:80240,0.350474208593369)
--(axis cs:80239,-0.86877053976059)
--(axis cs:80238,-2.20271110534668)
--(axis cs:80237,-3.45356512069702)
--(axis cs:80236,-4.37924432754517)
--(axis cs:80235,-4.92992448806763)
--(axis cs:80234,-5.09573364257812)
--(axis cs:80233,-4.79713153839111)
--(axis cs:80232,-4.20053672790527)
--cycle;

\addplot [semithick, steelblue31119180]
table {%
80136 -1.68819510936737
80137 -1.27418076992035
80138 -0.735438168048859
80139 -0.124704360961914
80140 0.498236626386642
80141 1.07240569591522
80142 1.54159820079803
80143 1.85988533496857
80144 1.99611032009125
80145 1.93693828582764
80146 1.68816149234772
80147 1.2741322517395
80148 0.735379755496979
80149 0.12464164942503
80150 -0.498297452926636
80151 -1.07245874404907
80152 -1.54163825511932
80153 -1.85990846157074
80154 -1.9961142539978
80155 -1.93692266941071
80156 -1.68812775611877
80157 -1.27408385276794
80158 -0.735321283340454
80159 -0.124578937888145
80160 0.498358309268951
80161 1.07251179218292
80162 1.54167830944061
80163 1.85993158817291
80164 1.99611818790436
80165 1.93690693378448
80166 1.68809401988983
80167 1.27403545379639
80168 0.735262870788574
80169 0.124516233801842
80170 -0.498419165611267
80171 -1.07256484031677
80172 -1.5417183637619
80173 -1.85995471477509
80174 -1.99612212181091
80175 -1.93689131736755
80176 -1.68806040287018
80177 -1.27398693561554
80178 -0.735204458236694
80179 -0.124453522264957
80180 0.498480021953583
80181 -4.44879961013794
80182 0.646496713161469
80183 3.18709683418274
80184 -4.1256308555603
80185 2.6250114440918
80186 -1.24886560440063
80187 -0.505231380462646
80188 0.73514598608017
80189 0.124390810728073
80190 -0.498540878295898
80191 -1.07267093658447
80192 -1.54179835319519
80193 -1.86000084877014
80194 -1.99612998962402
80195 -1.9368599653244
80196 -1.68799293041229
80197 -1.27389013767242
80198 -0.73508757352829
80199 -0.124328099191189
80200 0.498601704835892
80201 2.36293888092041
80202 0.505249261856079
80203 1.09504675865173
80204 0.362886697053909
80205 2.01431941986084
80206 4.22714805603027
80207 0.217120036482811
80208 0.73502916097641
80209 0.124265387654305
80210 -0.498662561178207
80211 -1.07277703285217
80212 -1.54187846183777
80213 -1.86004710197449
80214 -1.99613773822784
80215 -1.93682861328125
80216 -1.6879255771637
80217 -1.27379322052002
80218 -0.734970688819885
80219 -0.12420267611742
80220 0.498723417520523
80221 3.15210199356079
80222 -2.35510444641113
80223 -1.15109658241272
80224 3.28516864776611
80225 0.85689640045166
80226 -0.27378723025322
80227 3.67291069030762
80228 0.734912276268005
80229 0.372419893741608
80230 -1.49635279178619
80231 -3.21864914894104
};
\addplot [very thick, red, opacity=0.0018377001921336786]
table {%
80136 -6
80136 6
};
\addplot [very thick, red, opacity=0.0011683573952681462]
table {%
80137 -6
80137 6
};
\addplot [very thick, red, opacity=0.007762443312804798]
table {%
80138 -6
80138 6
};
\addplot [very thick, red, opacity=0.0]
table {%
80139 -6
80139 6
};
\addplot [very thick, red, opacity=0.006851895735432271]
table {%
80140 -6
80140 6
};
\addplot [very thick, red, opacity=0.0]
table {%
80141 -6
80141 6
};
\addplot [very thick, red, opacity=0.0]
table {%
80142 -6
80142 6
};
\addplot [very thick, red, opacity=0.01196625892174121]
table {%
80143 -6
80143 6
};
\addplot [very thick, red, opacity=0.0]
table {%
80144 -6
80144 6
};
\addplot [very thick, red, opacity=0.010807927503611417]
table {%
80145 -6
80145 6
};
\addplot [very thick, red, opacity=0.0]
table {%
80146 -6
80146 6
};
\addplot [very thick, red, opacity=6.495710451848467e-05]
table {%
80147 -6
80147 6
};
\addplot [very thick, red, opacity=0.0035274269542024807]
table {%
80148 -6
80148 6
};
\addplot [very thick, red, opacity=0.0]
table {%
80149 -6
80149 6
};
\addplot [very thick, red, opacity=0.0]
table {%
80150 -6
80150 6
};
\addplot [very thick, red, opacity=0.0]
table {%
80151 -6
80151 6
};
\addplot [very thick, red, opacity=0.00016168974822569735]
table {%
80152 -6
80152 6
};
\addplot [very thick, red, opacity=0.005621695160496014]
table {%
80153 -6
80153 6
};
\addplot [very thick, red, opacity=0.008593029306598322]
table {%
80154 -6
80154 6
};
\addplot [very thick, red, opacity=0.005726034345022765]
table {%
80155 -6
80155 6
};
\addplot [very thick, red, opacity=0.0009020338189173304]
table {%
80156 -6
80156 6
};
\addplot [very thick, red, opacity=0.0]
table {%
80157 -6
80157 6
};
\addplot [very thick, red, opacity=0.0]
table {%
80158 -6
80158 6
};
\addplot [very thick, red, opacity=0.0]
table {%
80159 -6
80159 6
};
\addplot [very thick, red, opacity=0.0011314570827912916]
table {%
80160 -6
80160 6
};
\addplot [very thick, red, opacity=0.0]
table {%
80161 -6
80161 6
};
\addplot [very thick, red, opacity=0.007335921499871348]
table {%
80162 -6
80162 6
};
\addplot [very thick, red, opacity=0.0]
table {%
80163 -6
80163 6
};
\addplot [very thick, red, opacity=0.005670402579701254]
table {%
80164 -6
80164 6
};
\addplot [very thick, red, opacity=0.0]
table {%
80165 -6
80165 6
};
\addplot [very thick, red, opacity=0.0]
table {%
80166 -6
80166 6
};
\addplot [very thick, red, opacity=0.00036480299212758637]
table {%
80167 -6
80167 6
};
\addplot [very thick, red, opacity=0.0009971314461086549]
table {%
80168 -6
80168 6
};
\addplot [very thick, red, opacity=0.0]
table {%
80169 -6
80169 6
};
\addplot [very thick, red, opacity=0.0014451324315836196]
table {%
80170 -6
80170 6
};
\addplot [very thick, red, opacity=0.0007969809713468616]
table {%
80171 -6
80171 6
};
\addplot [very thick, red, opacity=0.012980635712964806]
table {%
80172 -6
80172 6
};
\addplot [very thick, red, opacity=0.002131417180519712]
table {%
80173 -6
80173 6
};
\addplot [very thick, red, opacity=0.0006226580238866643]
table {%
80174 -6
80174 6
};
\addplot [very thick, red, opacity=0.015898499743837283]
table {%
80175 -6
80175 6
};
\addplot [very thick, red, opacity=0.0015267307812857152]
table {%
80176 -6
80176 6
};
\addplot [very thick, red, opacity=0.0]
table {%
80177 -6
80177 6
};
\addplot [very thick, red, opacity=0.0]
table {%
80178 -6
80178 6
};
\addplot [very thick, red, opacity=0.0004897172883592457]
table {%
80179 -6
80179 6
};
\addplot [very thick, red, opacity=0.004834109818994413]
table {%
80180 -6
80180 6
};
\addplot [very thick, red, opacity=0.0]
table {%
80181 -6
80181 6
};
\addplot [very thick, red, opacity=0.0]
table {%
80182 -6
80182 6
};
\addplot [very thick, red, opacity=0.0]
table {%
80183 -6
80183 6
};
\addplot [very thick, red, opacity=0.0018026511825408537]
table {%
80184 -6
80184 6
};
\addplot [very thick, red, opacity=0.0]
table {%
80185 -6
80185 6
};
\addplot [very thick, red, opacity=0.00029697791432344625]
table {%
80186 -6
80186 6
};
\addplot [very thick, red, opacity=0.0]
table {%
80187 -6
80187 6
};
\addplot [very thick, red, opacity=0.0025688936483502026]
table {%
80188 -6
80188 6
};
\addplot [very thick, red, opacity=9.300385849461867e-05]
table {%
80189 -6
80189 6
};
\addplot [very thick, red, opacity=0.0071581592443197145]
table {%
80190 -6
80190 6
};
\addplot [very thick, red, opacity=0.0]
table {%
80191 -6
80191 6
};
\addplot [very thick, red, opacity=0.0]
table {%
80192 -6
80192 6
};
\addplot [very thick, red, opacity=0.0037804222971920534]
table {%
80193 -6
80193 6
};
\addplot [very thick, red, opacity=0.0]
table {%
80194 -6
80194 6
};
\addplot [very thick, red, opacity=0.01431495373863732]
table {%
80195 -6
80195 6
};
\addplot [very thick, red, opacity=0.0]
table {%
80196 -6
80196 6
};
\addplot [very thick, red, opacity=0.023769530173721385]
table {%
80197 -6
80197 6
};
\addplot [very thick, red, opacity=0.0]
table {%
80198 -6
80198 6
};
\addplot [very thick, red, opacity=0.0017270753863499852]
table {%
80199 -6
80199 6
};
\addplot [very thick, red, opacity=0.0]
table {%
80200 -6
80200 6
};
\addplot [very thick, red, opacity=0.02324463958440848]
table {%
80201 -6
80201 6
};
\addplot [very thick, red, opacity=0.0]
table {%
80202 -6
80202 6
};
\addplot [very thick, red, opacity=0.009081128755659402]
table {%
80203 -6
80203 6
};
\addplot [very thick, red, opacity=0.0005993179389770438]
table {%
80204 -6
80204 6
};
\addplot [very thick, red, opacity=0.00032206466958103363]
table {%
80205 -6
80205 6
};
\addplot [very thick, red, opacity=0.0]
table {%
80206 -6
80206 6
};
\addplot [very thick, red, opacity=0.0]
table {%
80207 -6
80207 6
};
\addplot [very thick, red, opacity=0.0014352723227202529]
table {%
80208 -6
80208 6
};
\addplot [very thick, red, opacity=0.00013905603667588835]
table {%
80209 -6
80209 6
};
\addplot [very thick, red, opacity=0.00012268482600353406]
table {%
80210 -6
80210 6
};
\addplot [very thick, red, opacity=0.007171427857049348]
table {%
80211 -6
80211 6
};
\addplot [very thick, red, opacity=0.006391386100147161]
table {%
80212 -6
80212 6
};
\addplot [very thick, red, opacity=0.0]
table {%
80213 -6
80213 6
};
\addplot [very thick, red, opacity=0.0]
table {%
80214 -6
80214 6
};
\addplot [very thick, red, opacity=0.0]
table {%
80215 -6
80215 6
};
\addplot [very thick, red, opacity=0.0]
table {%
80216 -6
80216 6
};
\addplot [very thick, red, opacity=0.02146681943062102]
table {%
80217 -6
80217 6
};
\addplot [very thick, red, opacity=0.001029998282088411]
table {%
80218 -6
80218 6
};
\addplot [very thick, red, opacity=0.0]
table {%
80219 -6
80219 6
};
\addplot [very thick, red, opacity=0.0]
table {%
80220 -6
80220 6
};
\addplot [very thick, red, opacity=0.014473680666358008]
table {%
80221 -6
80221 6
};
\addplot [very thick, red, opacity=0.0]
table {%
80222 -6
80222 6
};
\addplot [very thick, red, opacity=0.0]
table {%
80223 -6
80223 6
};
\addplot [very thick, red, opacity=0.0027504339497445695]
table {%
80224 -6
80224 6
};
\addplot [very thick, red, opacity=0.0005646470363000105]
table {%
80225 -6
80225 6
};
\addplot [very thick, red, opacity=0.0]
table {%
80226 -6
80226 6
};
\addplot [very thick, red, opacity=0.0]
table {%
80227 -6
80227 6
};
\addplot [very thick, red, opacity=0.0005119315966023973]
table {%
80228 -6
80228 6
};
\addplot [very thick, red, opacity=0.0]
table {%
80229 -6
80229 6
};
\addplot [very thick, red, opacity=0.012642215967888023]
table {%
80230 -6
80230 6
};
\addplot [very thick, red, opacity=0.0]
table {%
80231 -6
80231 6
};
\addplot [very thick, blue, opacity=0.0]
table {%
80136 -6
80136 6
};
\addplot [very thick, blue, opacity=0.0]
table {%
80137 -6
80137 6
};
\addplot [very thick, blue, opacity=0.0]
table {%
80138 -6
80138 6
};
\addplot [very thick, blue, opacity=0.00014288433760912595]
table {%
80139 -6
80139 6
};
\addplot [very thick, blue, opacity=0.0]
table {%
80140 -6
80140 6
};
\addplot [very thick, blue, opacity=0.003235127829039185]
table {%
80141 -6
80141 6
};
\addplot [very thick, blue, opacity=0.0024982677051873774]
table {%
80142 -6
80142 6
};
\addplot [very thick, blue, opacity=0.0]
table {%
80143 -6
80143 6
};
\addplot [very thick, blue, opacity=0.005385799728322183]
table {%
80144 -6
80144 6
};
\addplot [very thick, blue, opacity=0.0]
table {%
80145 -6
80145 6
};
\addplot [very thick, blue, opacity=0.008513652274065362]
table {%
80146 -6
80146 6
};
\addplot [very thick, blue, opacity=0.0]
table {%
80147 -6
80147 6
};
\addplot [very thick, blue, opacity=0.0]
table {%
80148 -6
80148 6
};
\addplot [very thick, blue, opacity=0.0002090929691817992]
table {%
80149 -6
80149 6
};
\addplot [very thick, blue, opacity=0.0016631328511787487]
table {%
80150 -6
80150 6
};
\addplot [very thick, blue, opacity=0.0022409860760274343]
table {%
80151 -6
80151 6
};
\addplot [very thick, blue, opacity=0.0]
table {%
80152 -6
80152 6
};
\addplot [very thick, blue, opacity=0.0]
table {%
80153 -6
80153 6
};
\addplot [very thick, blue, opacity=0.0]
table {%
80154 -6
80154 6
};
\addplot [very thick, blue, opacity=0.0]
table {%
80155 -6
80155 6
};
\addplot [very thick, blue, opacity=0.0]
table {%
80156 -6
80156 6
};
\addplot [very thick, blue, opacity=0.0071213318451290595]
table {%
80157 -6
80157 6
};
\addplot [very thick, blue, opacity=0.001978311568171508]
table {%
80158 -6
80158 6
};
\addplot [very thick, blue, opacity=0.00019881726822035684]
table {%
80159 -6
80159 6
};
\addplot [very thick, blue, opacity=0.0]
table {%
80160 -6
80160 6
};
\addplot [very thick, blue, opacity=0.02179232025581364]
table {%
80161 -6
80161 6
};
\addplot [very thick, blue, opacity=0.0]
table {%
80162 -6
80162 6
};
\addplot [very thick, blue, opacity=0.003770852917826779]
table {%
80163 -6
80163 6
};
\addplot [very thick, blue, opacity=0.0]
table {%
80164 -6
80164 6
};
\addplot [very thick, blue, opacity=0.010378628841416443]
table {%
80165 -6
80165 6
};
\addplot [very thick, blue, opacity=0.0017945467428670905]
table {%
80166 -6
80166 6
};
\addplot [very thick, blue, opacity=0.0]
table {%
80167 -6
80167 6
};
\addplot [very thick, blue, opacity=0.0]
table {%
80168 -6
80168 6
};
\addplot [very thick, blue, opacity=0.0004390037870066066]
table {%
80169 -6
80169 6
};
\addplot [very thick, blue, opacity=0.0]
table {%
80170 -6
80170 6
};
\addplot [very thick, blue, opacity=0.0]
table {%
80171 -6
80171 6
};
\addplot [very thick, blue, opacity=0.0]
table {%
80172 -6
80172 6
};
\addplot [very thick, blue, opacity=0.0]
table {%
80173 -6
80173 6
};
\addplot [very thick, blue, opacity=0.0]
table {%
80174 -6
80174 6
};
\addplot [very thick, blue, opacity=0.0]
table {%
80175 -6
80175 6
};
\addplot [very thick, blue, opacity=0.0]
table {%
80176 -6
80176 6
};
\addplot [very thick, blue, opacity=0.016729872167941272]
table {%
80177 -6
80177 6
};
\addplot [very thick, blue, opacity=0.0004950983285741783]
table {%
80178 -6
80178 6
};
\addplot [very thick, blue, opacity=0.0]
table {%
80179 -6
80179 6
};
\addplot [very thick, blue, opacity=0.0]
table {%
80180 -6
80180 6
};
\addplot [very thick, blue, opacity=0.0036459319797759867]
table {%
80181 -6
80181 6
};
\addplot [very thick, blue, opacity=0.00027966862504878]
table {%
80182 -6
80182 6
};
\addplot [very thick, blue, opacity=0.021702868378459736]
table {%
80183 -6
80183 6
};
\addplot [very thick, blue, opacity=0.0]
table {%
80184 -6
80184 6
};
\addplot [very thick, blue, opacity=0.01065200195274768]
table {%
80185 -6
80185 6
};
\addplot [very thick, blue, opacity=0.0]
table {%
80186 -6
80186 6
};
\addplot [very thick, blue, opacity=0.004512772619817785]
table {%
80187 -6
80187 6
};
\addplot [very thick, blue, opacity=0.0]
table {%
80188 -6
80188 6
};
\addplot [very thick, blue, opacity=0.0]
table {%
80189 -6
80189 6
};
\addplot [very thick, blue, opacity=0.0]
table {%
80190 -6
80190 6
};
\addplot [very thick, blue, opacity=0.0024155365491998]
table {%
80191 -6
80191 6
};
\addplot [very thick, blue, opacity=0.000560140561911048]
table {%
80192 -6
80192 6
};
\addplot [very thick, blue, opacity=0.0]
table {%
80193 -6
80193 6
};
\addplot [very thick, blue, opacity=0.000390184235187087]
table {%
80194 -6
80194 6
};
\addplot [very thick, blue, opacity=0.0]
table {%
80195 -6
80195 6
};
\addplot [very thick, blue, opacity=0.050356259266447455]
table {%
80196 -6
80196 6
};
\addplot [very thick, blue, opacity=0.0]
table {%
80197 -6
80197 6
};
\addplot [very thick, blue, opacity=0.009615039551818222]
table {%
80198 -6
80198 6
};
\addplot [very thick, blue, opacity=0.0]
table {%
80199 -6
80199 6
};
\addplot [very thick, blue, opacity=0.0006513625704096846]
table {%
80200 -6
80200 6
};
\addplot [very thick, blue, opacity=0.0]
table {%
80201 -6
80201 6
};
\addplot [very thick, blue, opacity=0.0008790287669758264]
table {%
80202 -6
80202 6
};
\addplot [very thick, blue, opacity=0.0]
table {%
80203 -6
80203 6
};
\addplot [very thick, blue, opacity=0.0]
table {%
80204 -6
80204 6
};
\addplot [very thick, blue, opacity=0.0]
table {%
80205 -6
80205 6
};
\addplot [very thick, blue, opacity=0.015839835486687416]
table {%
80206 -6
80206 6
};
\addplot [very thick, blue, opacity=8.177639359633434e-05]
table {%
80207 -6
80207 6
};
\addplot [very thick, blue, opacity=0.0]
table {%
80208 -6
80208 6
};
\addplot [very thick, blue, opacity=0.0]
table {%
80209 -6
80209 6
};
\addplot [very thick, blue, opacity=0.0]
table {%
80210 -6
80210 6
};
\addplot [very thick, blue, opacity=0.0]
table {%
80211 -6
80211 6
};
\addplot [very thick, blue, opacity=0.0]
table {%
80212 -6
80212 6
};
\addplot [very thick, blue, opacity=0.021911668007389813]
table {%
80213 -6
80213 6
};
\addplot [very thick, blue, opacity=0.0001979564307202131]
table {%
80214 -6
80214 6
};
\addplot [very thick, blue, opacity=0.039651290767074895]
table {%
80215 -6
80215 6
};
\addplot [very thick, blue, opacity=0.02998509121538521]
table {%
80216 -6
80216 6
};
\addplot [very thick, blue, opacity=0.0]
table {%
80217 -6
80217 6
};
\addplot [very thick, blue, opacity=0.0]
table {%
80218 -6
80218 6
};
\addplot [very thick, blue, opacity=2.4202959350805003e-05]
table {%
80219 -6
80219 6
};
\addplot [very thick, blue, opacity=0.0011668832669704708]
table {%
80220 -6
80220 6
};
\addplot [very thick, blue, opacity=0.0]
table {%
80221 -6
80221 6
};
\addplot [very thick, blue, opacity=0.017383412765608047]
table {%
80222 -6
80222 6
};
\addplot [very thick, blue, opacity=0.0006556866647375831]
table {%
80223 -6
80223 6
};
\addplot [very thick, blue, opacity=0.0]
table {%
80224 -6
80224 6
};
\addplot [very thick, blue, opacity=0.0]
table {%
80225 -6
80225 6
};
\addplot [very thick, blue, opacity=0.00024475272315059067]
table {%
80226 -6
80226 6
};
\addplot [very thick, blue, opacity=0.013289800879341854]
table {%
80227 -6
80227 6
};
\addplot [very thick, blue, opacity=0.0]
table {%
80228 -6
80228 6
};
\addplot [very thick, blue, opacity=0.015090090293100655]
table {%
80229 -6
80229 6
};
\addplot [very thick, blue, opacity=0.0]
table {%
80230 -6
80230 6
};
\addplot [very thick, blue, opacity=1.0]
table {%
80231 -6
80231 6
};
\addplot [semithick, green]
table {%
80232 -4.22009515762329
80233 -4.85167217254639
80234 -5.16187906265259
80235 -4.9884786605835
80236 -4.48050212860107
80237 -3.52439045906067
80238 -2.24922728538513
80239 -0.93678867816925
80240 0.255079239606857
80241 1.29498779773712
80242 2.04991245269775
80243 2.56430172920227
80244 2.89131212234497
80245 2.90039682388306
80246 2.60961294174194
80247 1.84990000724792
80248 0.931938529014587
80249 -0.0701910406351089
80250 -1.23361730575562
80251 -2.33042287826538
80252 -3.19561100006104
80253 -3.90092897415161
80254 -4.23897218704224
80255 -4.10508489608765
80256 -3.69425892829895
80257 -2.97898697853088
80258 -2.1272988319397
80259 -1.14675402641296
80260 -0.264372140169144
80261 0.538671493530273
80262 1.02260315418243
80263 1.39976096153259
80264 1.52306413650513
80265 1.37416398525238
80266 1.06836700439453
80267 0.625422477722168
80268 0.027860626578331
80269 -0.612323939800262
80270 -1.21768629550934
80271 -1.71643507480621
80272 -2.15612363815308
80273 -2.43774747848511
80274 -2.61007404327393
80275 -2.55597114562988
80276 -2.26129913330078
80277 -1.83636748790741
80278 -1.35848164558411
80279 -0.585540533065796
};
\addplot [semithick, green, mark=x, mark size=5, mark options={solid}]
table {%
80277 -1.83636748790741
};

\nextgroupplot[
scaled x ticks=manual:{}{\pgfmathparse{#1}},
tick align=outside,
tick pos=left,
title={sine\_spiky\_a},
x grid style={darkgray176},
xmin=80128.85, xmax=80286.15,
xtick style={color=black},
xticklabels={},
y grid style={darkgray176},
ymin=-6, ymax=6,
ytick style={color=black}
]
\addplot [semithick, steelblue31119180]
table {%
80136 -1.68819510936737
80137 -1.27418076992035
80138 -0.735438168048859
80139 -0.124704360961914
80140 0.498236626386642
80141 1.07240569591522
80142 1.54159820079803
80143 1.85988533496857
80144 -3.00388956069946
80145 1.93693828582764
80146 1.68816149234772
80147 1.2741322517395
80148 0.735379755496979
80149 0.12464164942503
80150 -0.498297452926636
80151 -1.07245874404907
80152 -1.54163825511932
80153 -1.85990846157074
80154 -1.9961142539978
80155 -1.93692266941071
80156 -1.68812775611877
80157 -1.27408385276794
80158 -0.735321283340454
80159 -0.124578937888145
80160 0.498358309268951
80161 1.07251179218292
80162 1.54167830944061
80163 1.85993158817291
80164 -3.00388169288635
80165 1.93690693378448
80166 1.68809401988983
80167 1.27403545379639
80168 0.735262870788574
80169 0.124516233801842
80170 -0.498419165611267
80171 -1.07256484031677
80172 -1.5417183637619
80173 -1.85995471477509
80174 -1.99612212181091
80175 -1.93689131736755
80176 -1.68806040287018
80177 -1.27398693561554
80178 -0.735204458236694
80179 -0.124453522264957
80180 0.498480021953583
80181 1.07261788845062
80182 1.54175841808319
80183 1.85997784137726
80184 -3.00387406349182
80185 1.93687570095062
80186 1.68802666664124
80187 1.27393853664398
80188 0.73514598608017
80189 0.373172432184219
80190 -1.4956226348877
80191 -3.21801280975342
80192 -4.62539529800415
80193 -5.580002784729
80194 -5.98838996887207
80195 -5.81057977676392
80196 -5.06397867202759
80197 -3.82167029380798
80198 -2.20526266098022
80199 -0.372984290122986
80200 1.49580514431
80201 3.21817183494568
80202 4.62551498413086
80203 5.58007192611694
80204 5.98840141296387
80205 5.81053304672241
80206 5.06387758255005
80207 3.82152509689331
80208 2.20508742332458
80209 0.124265387654305
80210 -0.498662561178207
80211 -1.07277703285217
80212 -1.54187846183777
80213 -1.86004710197449
80214 -1.99613773822784
80215 -1.93682861328125
80216 -1.6879255771637
80217 -1.27379322052002
80218 -0.734970688819885
80219 -0.12420267611742
80220 0.498723417520523
80221 1.07282996177673
80222 1.54191839694977
80223 1.86007010936737
80224 1.9961416721344
80225 1.93681299686432
80226 1.68789184093475
80227 1.27374482154846
80228 0.734912276268005
80229 0.124139964580536
80230 -0.498784244060516
80231 -1.07288300991058
};
\addplot [very thick, red, opacity=0.0]
table {%
80136 -6
80136 6
};
\addplot [very thick, red, opacity=0.0]
table {%
80137 -6
80137 6
};
\addplot [very thick, red, opacity=0.0]
table {%
80138 -6
80138 6
};
\addplot [very thick, red, opacity=0.0]
table {%
80139 -6
80139 6
};
\addplot [very thick, red, opacity=0.0011777085323272758]
table {%
80140 -6
80140 6
};
\addplot [very thick, red, opacity=0.0]
table {%
80141 -6
80141 6
};
\addplot [very thick, red, opacity=0.00278267327255962]
table {%
80142 -6
80142 6
};
\addplot [very thick, red, opacity=0.023551309345250425]
table {%
80143 -6
80143 6
};
\addplot [very thick, red, opacity=0.0]
table {%
80144 -6
80144 6
};
\addplot [very thick, red, opacity=0.01947852660413777]
table {%
80145 -6
80145 6
};
\addplot [very thick, red, opacity=0.0284593862975341]
table {%
80146 -6
80146 6
};
\addplot [very thick, red, opacity=0.0]
table {%
80147 -6
80147 6
};
\addplot [very thick, red, opacity=0.0]
table {%
80148 -6
80148 6
};
\addplot [very thick, red, opacity=0.0]
table {%
80149 -6
80149 6
};
\addplot [very thick, red, opacity=0.0]
table {%
80150 -6
80150 6
};
\addplot [very thick, red, opacity=0.001648627967646468]
table {%
80151 -6
80151 6
};
\addplot [very thick, red, opacity=0.0]
table {%
80152 -6
80152 6
};
\addplot [very thick, red, opacity=0.0]
table {%
80153 -6
80153 6
};
\addplot [very thick, red, opacity=0.0009320354127431515]
table {%
80154 -6
80154 6
};
\addplot [very thick, red, opacity=0.0]
table {%
80155 -6
80155 6
};
\addplot [very thick, red, opacity=0.0]
table {%
80156 -6
80156 6
};
\addplot [very thick, red, opacity=0.02027234170196028]
table {%
80157 -6
80157 6
};
\addplot [very thick, red, opacity=0.0]
table {%
80158 -6
80158 6
};
\addplot [very thick, red, opacity=0.0]
table {%
80159 -6
80159 6
};
\addplot [very thick, red, opacity=0.00044296708788851575]
table {%
80160 -6
80160 6
};
\addplot [very thick, red, opacity=0.0]
table {%
80161 -6
80161 6
};
\addplot [very thick, red, opacity=0.0]
table {%
80162 -6
80162 6
};
\addplot [very thick, red, opacity=0.0]
table {%
80163 -6
80163 6
};
\addplot [very thick, red, opacity=0.0]
table {%
80164 -6
80164 6
};
\addplot [very thick, red, opacity=0.00985310608655425]
table {%
80165 -6
80165 6
};
\addplot [very thick, red, opacity=0.03733540979443924]
table {%
80166 -6
80166 6
};
\addplot [very thick, red, opacity=0.0]
table {%
80167 -6
80167 6
};
\addplot [very thick, red, opacity=0.0005178968744167302]
table {%
80168 -6
80168 6
};
\addplot [very thick, red, opacity=0.0]
table {%
80169 -6
80169 6
};
\addplot [very thick, red, opacity=0.0]
table {%
80170 -6
80170 6
};
\addplot [very thick, red, opacity=0.0011630491662381233]
table {%
80171 -6
80171 6
};
\addplot [very thick, red, opacity=0.0]
table {%
80172 -6
80172 6
};
\addplot [very thick, red, opacity=0.002903518632287807]
table {%
80173 -6
80173 6
};
\addplot [very thick, red, opacity=0.0]
table {%
80174 -6
80174 6
};
\addplot [very thick, red, opacity=0.02787440851252262]
table {%
80175 -6
80175 6
};
\addplot [very thick, red, opacity=7.123077855100342e-05]
table {%
80176 -6
80176 6
};
\addplot [very thick, red, opacity=0.0]
table {%
80177 -6
80177 6
};
\addplot [very thick, red, opacity=0.0]
table {%
80178 -6
80178 6
};
\addplot [very thick, red, opacity=0.0]
table {%
80179 -6
80179 6
};
\addplot [very thick, red, opacity=0.0014341058649625686]
table {%
80180 -6
80180 6
};
\addplot [very thick, red, opacity=0.07477984569458823]
table {%
80181 -6
80181 6
};
\addplot [very thick, red, opacity=0.004339679095194049]
table {%
80182 -6
80182 6
};
\addplot [very thick, red, opacity=0.1099387624410709]
table {%
80183 -6
80183 6
};
\addplot [very thick, red, opacity=0.0]
table {%
80184 -6
80184 6
};
\addplot [very thick, red, opacity=0.09927087495892056]
table {%
80185 -6
80185 6
};
\addplot [very thick, red, opacity=0.04493663876613608]
table {%
80186 -6
80186 6
};
\addplot [very thick, red, opacity=0.0]
table {%
80187 -6
80187 6
};
\addplot [very thick, red, opacity=0.0]
table {%
80188 -6
80188 6
};
\addplot [very thick, red, opacity=0.0]
table {%
80189 -6
80189 6
};
\addplot [very thick, red, opacity=0.19393733345269548]
table {%
80190 -6
80190 6
};
\addplot [very thick, red, opacity=0.024833550273827967]
table {%
80191 -6
80191 6
};
\addplot [very thick, red, opacity=0.5116120673451008]
table {%
80192 -6
80192 6
};
\addplot [very thick, red, opacity=0.006958513474862337]
table {%
80193 -6
80193 6
};
\addplot [very thick, red, opacity=0.0]
table {%
80194 -6
80194 6
};
\addplot [very thick, red, opacity=0.25444723078926007]
table {%
80195 -6
80195 6
};
\addplot [very thick, red, opacity=0.0]
table {%
80196 -6
80196 6
};
\addplot [very thick, red, opacity=0.0]
table {%
80197 -6
80197 6
};
\addplot [very thick, red, opacity=0.0]
table {%
80198 -6
80198 6
};
\addplot [very thick, red, opacity=0.0]
table {%
80199 -6
80199 6
};
\addplot [very thick, red, opacity=0.010990683118049835]
table {%
80200 -6
80200 6
};
\addplot [very thick, red, opacity=0.0]
table {%
80201 -6
80201 6
};
\addplot [very thick, red, opacity=0.012211724651806985]
table {%
80202 -6
80202 6
};
\addplot [very thick, red, opacity=0.015102166008177997]
table {%
80203 -6
80203 6
};
\addplot [very thick, red, opacity=0.06339501969178074]
table {%
80204 -6
80204 6
};
\addplot [very thick, red, opacity=0.0]
table {%
80205 -6
80205 6
};
\addplot [very thick, red, opacity=0.0]
table {%
80206 -6
80206 6
};
\addplot [very thick, red, opacity=0.0]
table {%
80207 -6
80207 6
};
\addplot [very thick, red, opacity=0.0]
table {%
80208 -6
80208 6
};
\addplot [very thick, red, opacity=0.0]
table {%
80209 -6
80209 6
};
\addplot [very thick, red, opacity=0.008159613097563781]
table {%
80210 -6
80210 6
};
\addplot [very thick, red, opacity=0.004363521647304146]
table {%
80211 -6
80211 6
};
\addplot [very thick, red, opacity=0.0]
table {%
80212 -6
80212 6
};
\addplot [very thick, red, opacity=0.0010565649149507897]
table {%
80213 -6
80213 6
};
\addplot [very thick, red, opacity=0.0]
table {%
80214 -6
80214 6
};
\addplot [very thick, red, opacity=0.030146721174911822]
table {%
80215 -6
80215 6
};
\addplot [very thick, red, opacity=0.003308934968695273]
table {%
80216 -6
80216 6
};
\addplot [very thick, red, opacity=0.02295200560790072]
table {%
80217 -6
80217 6
};
\addplot [very thick, red, opacity=0.0014944414197253016]
table {%
80218 -6
80218 6
};
\addplot [very thick, red, opacity=0.0]
table {%
80219 -6
80219 6
};
\addplot [very thick, red, opacity=0.0]
table {%
80220 -6
80220 6
};
\addplot [very thick, red, opacity=0.0]
table {%
80221 -6
80221 6
};
\addplot [very thick, red, opacity=0.0]
table {%
80222 -6
80222 6
};
\addplot [very thick, red, opacity=0.0]
table {%
80223 -6
80223 6
};
\addplot [very thick, red, opacity=0.0]
table {%
80224 -6
80224 6
};
\addplot [very thick, red, opacity=0.0]
table {%
80225 -6
80225 6
};
\addplot [very thick, red, opacity=0.0]
table {%
80226 -6
80226 6
};
\addplot [very thick, red, opacity=0.0]
table {%
80227 -6
80227 6
};
\addplot [very thick, red, opacity=0.0]
table {%
80228 -6
80228 6
};
\addplot [very thick, red, opacity=0.00020057306907704178]
table {%
80229 -6
80229 6
};
\addplot [very thick, red, opacity=0.0]
table {%
80230 -6
80230 6
};
\addplot [very thick, red, opacity=0.0]
table {%
80231 -6
80231 6
};
\addplot [very thick, blue, opacity=0.0011045914706544024]
table {%
80136 -6
80136 6
};
\addplot [very thick, blue, opacity=0.011135000809747605]
table {%
80137 -6
80137 6
};
\addplot [very thick, blue, opacity=0.0018017647047583365]
table {%
80138 -6
80138 6
};
\addplot [very thick, blue, opacity=0.0003004852676234149]
table {%
80139 -6
80139 6
};
\addplot [very thick, blue, opacity=0.0]
table {%
80140 -6
80140 6
};
\addplot [very thick, blue, opacity=0.001633082229870145]
table {%
80141 -6
80141 6
};
\addplot [very thick, blue, opacity=0.0]
table {%
80142 -6
80142 6
};
\addplot [very thick, blue, opacity=0.0]
table {%
80143 -6
80143 6
};
\addplot [very thick, blue, opacity=0.004366175835173442]
table {%
80144 -6
80144 6
};
\addplot [very thick, blue, opacity=0.0]
table {%
80145 -6
80145 6
};
\addplot [very thick, blue, opacity=0.0]
table {%
80146 -6
80146 6
};
\addplot [very thick, blue, opacity=0.0023363996025506863]
table {%
80147 -6
80147 6
};
\addplot [very thick, blue, opacity=0.01937539930235598]
table {%
80148 -6
80148 6
};
\addplot [very thick, blue, opacity=0.00037165495952960916]
table {%
80149 -6
80149 6
};
\addplot [very thick, blue, opacity=0.0019065435386317487]
table {%
80150 -6
80150 6
};
\addplot [very thick, blue, opacity=0.0]
table {%
80151 -6
80151 6
};
\addplot [very thick, blue, opacity=0.0482394381252647]
table {%
80152 -6
80152 6
};
\addplot [very thick, blue, opacity=0.00480366331490434]
table {%
80153 -6
80153 6
};
\addplot [very thick, blue, opacity=0.0]
table {%
80154 -6
80154 6
};
\addplot [very thick, blue, opacity=0.034592131293278705]
table {%
80155 -6
80155 6
};
\addplot [very thick, blue, opacity=0.005850904925366853]
table {%
80156 -6
80156 6
};
\addplot [very thick, blue, opacity=0.0]
table {%
80157 -6
80157 6
};
\addplot [very thick, blue, opacity=0.0015974156150609813]
table {%
80158 -6
80158 6
};
\addplot [very thick, blue, opacity=0.003158113173762211]
table {%
80159 -6
80159 6
};
\addplot [very thick, blue, opacity=0.0]
table {%
80160 -6
80160 6
};
\addplot [very thick, blue, opacity=0.004031487235551734]
table {%
80161 -6
80161 6
};
\addplot [very thick, blue, opacity=8.934349590722861e-05]
table {%
80162 -6
80162 6
};
\addplot [very thick, blue, opacity=0.00966155954631631]
table {%
80163 -6
80163 6
};
\addplot [very thick, blue, opacity=0.006958728664219447]
table {%
80164 -6
80164 6
};
\addplot [very thick, blue, opacity=0.0]
table {%
80165 -6
80165 6
};
\addplot [very thick, blue, opacity=0.0]
table {%
80166 -6
80166 6
};
\addplot [very thick, blue, opacity=0.002991705465581076]
table {%
80167 -6
80167 6
};
\addplot [very thick, blue, opacity=0.0]
table {%
80168 -6
80168 6
};
\addplot [very thick, blue, opacity=0.00028308433883955924]
table {%
80169 -6
80169 6
};
\addplot [very thick, blue, opacity=0.014683992713192274]
table {%
80170 -6
80170 6
};
\addplot [very thick, blue, opacity=0.0]
table {%
80171 -6
80171 6
};
\addplot [very thick, blue, opacity=0.012797281549379475]
table {%
80172 -6
80172 6
};
\addplot [very thick, blue, opacity=0.0]
table {%
80173 -6
80173 6
};
\addplot [very thick, blue, opacity=0.0009492001214557714]
table {%
80174 -6
80174 6
};
\addplot [very thick, blue, opacity=0.0]
table {%
80175 -6
80175 6
};
\addplot [very thick, blue, opacity=0.0]
table {%
80176 -6
80176 6
};
\addplot [very thick, blue, opacity=0.02660096148876211]
table {%
80177 -6
80177 6
};
\addplot [very thick, blue, opacity=0.002862123716276831]
table {%
80178 -6
80178 6
};
\addplot [very thick, blue, opacity=0.005082312692466498]
table {%
80179 -6
80179 6
};
\addplot [very thick, blue, opacity=0.0]
table {%
80180 -6
80180 6
};
\addplot [very thick, blue, opacity=0.0]
table {%
80181 -6
80181 6
};
\addplot [very thick, blue, opacity=0.0]
table {%
80182 -6
80182 6
};
\addplot [very thick, blue, opacity=0.0]
table {%
80183 -6
80183 6
};
\addplot [very thick, blue, opacity=0.0034948507564233787]
table {%
80184 -6
80184 6
};
\addplot [very thick, blue, opacity=0.0]
table {%
80185 -6
80185 6
};
\addplot [very thick, blue, opacity=0.0]
table {%
80186 -6
80186 6
};
\addplot [very thick, blue, opacity=0.0028081880386456887]
table {%
80187 -6
80187 6
};
\addplot [very thick, blue, opacity=0.02517078216142322]
table {%
80188 -6
80188 6
};
\addplot [very thick, blue, opacity=0.0024559980942454106]
table {%
80189 -6
80189 6
};
\addplot [very thick, blue, opacity=0.0]
table {%
80190 -6
80190 6
};
\addplot [very thick, blue, opacity=0.0]
table {%
80191 -6
80191 6
};
\addplot [very thick, blue, opacity=0.0]
table {%
80192 -6
80192 6
};
\addplot [very thick, blue, opacity=0.0]
table {%
80193 -6
80193 6
};
\addplot [very thick, blue, opacity=0.0008978798542764275]
table {%
80194 -6
80194 6
};
\addplot [very thick, blue, opacity=0.0]
table {%
80195 -6
80195 6
};
\addplot [very thick, blue, opacity=0.029774667883504607]
table {%
80196 -6
80196 6
};
\addplot [very thick, blue, opacity=0.23193005709548617]
table {%
80197 -6
80197 6
};
\addplot [very thick, blue, opacity=0.020999735478337767]
table {%
80198 -6
80198 6
};
\addplot [very thick, blue, opacity=0.028806256305754167]
table {%
80199 -6
80199 6
};
\addplot [very thick, blue, opacity=0.0]
table {%
80200 -6
80200 6
};
\addplot [very thick, blue, opacity=0.05511888818254676]
table {%
80201 -6
80201 6
};
\addplot [very thick, blue, opacity=0.0]
table {%
80202 -6
80202 6
};
\addplot [very thick, blue, opacity=0.0]
table {%
80203 -6
80203 6
};
\addplot [very thick, blue, opacity=0.0]
table {%
80204 -6
80204 6
};
\addplot [very thick, blue, opacity=0.01665352707575827]
table {%
80205 -6
80205 6
};
\addplot [very thick, blue, opacity=0.15228178128809847]
table {%
80206 -6
80206 6
};
\addplot [very thick, blue, opacity=0.0088138447757611]
table {%
80207 -6
80207 6
};
\addplot [very thick, blue, opacity=0.03072806040245781]
table {%
80208 -6
80208 6
};
\addplot [very thick, blue, opacity=0.0005335632025351438]
table {%
80209 -6
80209 6
};
\addplot [very thick, blue, opacity=0.0]
table {%
80210 -6
80210 6
};
\addplot [very thick, blue, opacity=0.0]
table {%
80211 -6
80211 6
};
\addplot [very thick, blue, opacity=0.029283956376354368]
table {%
80212 -6
80212 6
};
\addplot [very thick, blue, opacity=0.0]
table {%
80213 -6
80213 6
};
\addplot [very thick, blue, opacity=0.0023614359174531528]
table {%
80214 -6
80214 6
};
\addplot [very thick, blue, opacity=0.0]
table {%
80215 -6
80215 6
};
\addplot [very thick, blue, opacity=0.0]
table {%
80216 -6
80216 6
};
\addplot [very thick, blue, opacity=0.0]
table {%
80217 -6
80217 6
};
\addplot [very thick, blue, opacity=0.0]
table {%
80218 -6
80218 6
};
\addplot [very thick, blue, opacity=0.0010525961272837623]
table {%
80219 -6
80219 6
};
\addplot [very thick, blue, opacity=0.0024788494318280876]
table {%
80220 -6
80220 6
};
\addplot [very thick, blue, opacity=0.05642640146866626]
table {%
80221 -6
80221 6
};
\addplot [very thick, blue, opacity=0.013770989843276365]
table {%
80222 -6
80222 6
};
\addplot [very thick, blue, opacity=0.17202464487203223]
table {%
80223 -6
80223 6
};
\addplot [very thick, blue, opacity=0.21430930272632787]
table {%
80224 -6
80224 6
};
\addplot [very thick, blue, opacity=0.014010158368458871]
table {%
80225 -6
80225 6
};
\addplot [very thick, blue, opacity=0.285001242613525]
table {%
80226 -6
80226 6
};
\addplot [very thick, blue, opacity=0.0035719292727201843]
table {%
80227 -6
80227 6
};
\addplot [very thick, blue, opacity=0.06830255944431082]
table {%
80228 -6
80228 6
};
\addplot [very thick, blue, opacity=0.0]
table {%
80229 -6
80229 6
};
\addplot [very thick, blue, opacity=0.0034096621341511623]
table {%
80230 -6
80230 6
};
\addplot [very thick, blue, opacity=0.028700095246310908]
table {%
80231 -6
80231 6
};

\nextgroupplot[
tick align=outside,
tick pos=left,
title={sine\_spiky\_b},
x grid style={darkgray176},
xmin=80128.85, xmax=80286.15,
xtick style={color=black},
y grid style={darkgray176},
ymin=-6, ymax=6,
ytick style={color=black}
]
\addplot [semithick, steelblue31119180]
table {%
80136 -1.68819510936737
80137 -1.27418076992035
80138 -0.735438168048859
80139 -0.124704360961914
80140 0.498236626386642
80141 1.07240569591522
80142 1.54159820079803
80143 1.85988533496857
80144 1.99611032009125
80145 1.93693828582764
80146 1.68816149234772
80147 1.2741322517395
80148 0.735379755496979
80149 0.12464164942503
80150 -0.498297452926636
80151 -1.07245874404907
80152 -1.54163825511932
80153 -1.85990846157074
80154 -1.9961142539978
80155 -1.93692266941071
80156 -1.68812775611877
80157 -1.27408385276794
80158 -0.735321283340454
80159 -0.124578937888145
80160 0.498358309268951
80161 1.07251179218292
80162 1.54167830944061
80163 1.85993158817291
80164 1.99611818790436
80165 1.93690693378448
80166 1.68809401988983
80167 1.27403545379639
80168 0.735262870788574
80169 0.124516233801842
80170 -0.498419165611267
80171 -1.07256484031677
80172 -1.5417183637619
80173 -1.85995471477509
80174 -1.99612212181091
80175 -1.93689131736755
80176 -1.68806040287018
80177 -1.27398693561554
80178 -0.735204458236694
80179 -0.124453522264957
80180 0.498480021953583
80181 1.07261788845062
80182 1.54175841808319
80183 1.85997784137726
80184 1.99612605571747
80185 1.93687570095062
80186 1.68802666664124
80187 1.27393853664398
80188 0.73514598608017
80189 0.124390810728073
80190 -0.498540878295898
80191 -1.07267093658447
80192 -1.54179835319519
80193 -1.86000084877014
80194 -1.99612998962402
80195 -1.9368599653244
80196 -1.68799293041229
80197 -1.27389013767242
80198 -0.73508757352829
80199 -0.124328099191189
80200 0.498601704835892
80201 1.07272398471832
80202 1.54183840751648
80203 1.86002397537231
80204 1.99613380432129
80205 1.93684434890747
80206 1.68795931339264
80207 1.27384173870087
80208 0.73502916097641
80209 0.124265387654305
80210 -0.498662561178207
80211 -1.07277703285217
80212 -1.54187846183777
80213 -1.86004710197449
80214 -1.99613773822784
80215 -1.93682861328125
80216 -1.6879255771637
80217 -1.27379322052002
80218 -0.734970688819885
80219 -0.12420267611742
80220 0.498723417520523
80221 1.07282996177673
80222 1.54191839694977
80223 1.86007010936737
80224 -3.0038583278656
80225 1.93681299686432
80226 1.68789184093475
80227 1.27374482154846
80228 0.734912276268005
80229 0.124139964580536
80230 -0.498784244060516
80231 -1.07288300991058
};
\addplot [very thick, red, opacity=0.0]
table {%
80136 -6
80136 6
};
\addplot [very thick, red, opacity=0.0]
table {%
80137 -6
80137 6
};
\addplot [very thick, red, opacity=0.0]
table {%
80138 -6
80138 6
};
\addplot [very thick, red, opacity=0.0]
table {%
80139 -6
80139 6
};
\addplot [very thick, red, opacity=0.0004903917356704826]
table {%
80140 -6
80140 6
};
\addplot [very thick, red, opacity=0.0]
table {%
80141 -6
80141 6
};
\addplot [very thick, red, opacity=0.0013556841171188114]
table {%
80142 -6
80142 6
};
\addplot [very thick, red, opacity=0.0]
table {%
80143 -6
80143 6
};
\addplot [very thick, red, opacity=0.007058183409898825]
table {%
80144 -6
80144 6
};
\addplot [very thick, red, opacity=0.0]
table {%
80145 -6
80145 6
};
\addplot [very thick, red, opacity=0.0030034490752389653]
table {%
80146 -6
80146 6
};
\addplot [very thick, red, opacity=0.0]
table {%
80147 -6
80147 6
};
\addplot [very thick, red, opacity=0.0]
table {%
80148 -6
80148 6
};
\addplot [very thick, red, opacity=0.0]
table {%
80149 -6
80149 6
};
\addplot [very thick, red, opacity=0.0024623345446949016]
table {%
80150 -6
80150 6
};
\addplot [very thick, red, opacity=0.0]
table {%
80151 -6
80151 6
};
\addplot [very thick, red, opacity=0.03674543629569404]
table {%
80152 -6
80152 6
};
\addplot [very thick, red, opacity=0.0]
table {%
80153 -6
80153 6
};
\addplot [very thick, red, opacity=0.0]
table {%
80154 -6
80154 6
};
\addplot [very thick, red, opacity=0.0]
table {%
80155 -6
80155 6
};
\addplot [very thick, red, opacity=0.0]
table {%
80156 -6
80156 6
};
\addplot [very thick, red, opacity=0.03334169053171152]
table {%
80157 -6
80157 6
};
\addplot [very thick, red, opacity=0.0]
table {%
80158 -6
80158 6
};
\addplot [very thick, red, opacity=0.0]
table {%
80159 -6
80159 6
};
\addplot [very thick, red, opacity=0.0007603095684968846]
table {%
80160 -6
80160 6
};
\addplot [very thick, red, opacity=0.0]
table {%
80161 -6
80161 6
};
\addplot [very thick, red, opacity=0.0019757816077978665]
table {%
80162 -6
80162 6
};
\addplot [very thick, red, opacity=0.0]
table {%
80163 -6
80163 6
};
\addplot [very thick, red, opacity=0.0599895955364733]
table {%
80164 -6
80164 6
};
\addplot [very thick, red, opacity=0.0]
table {%
80165 -6
80165 6
};
\addplot [very thick, red, opacity=0.03898846687212747]
table {%
80166 -6
80166 6
};
\addplot [very thick, red, opacity=0.0]
table {%
80167 -6
80167 6
};
\addplot [very thick, red, opacity=0.014651311426238448]
table {%
80168 -6
80168 6
};
\addplot [very thick, red, opacity=0.0]
table {%
80169 -6
80169 6
};
\addplot [very thick, red, opacity=0.0]
table {%
80170 -6
80170 6
};
\addplot [very thick, red, opacity=0.002309964840654545]
table {%
80171 -6
80171 6
};
\addplot [very thick, red, opacity=0.06343802659644761]
table {%
80172 -6
80172 6
};
\addplot [very thick, red, opacity=0.0]
table {%
80173 -6
80173 6
};
\addplot [very thick, red, opacity=0.0]
table {%
80174 -6
80174 6
};
\addplot [very thick, red, opacity=0.0]
table {%
80175 -6
80175 6
};
\addplot [very thick, red, opacity=0.0]
table {%
80176 -6
80176 6
};
\addplot [very thick, red, opacity=0.0]
table {%
80177 -6
80177 6
};
\addplot [very thick, red, opacity=0.0]
table {%
80178 -6
80178 6
};
\addplot [very thick, red, opacity=0.0]
table {%
80179 -6
80179 6
};
\addplot [very thick, red, opacity=0.0017178364728298442]
table {%
80180 -6
80180 6
};
\addplot [very thick, red, opacity=0.10848934849160086]
table {%
80181 -6
80181 6
};
\addplot [very thick, red, opacity=0.005663943129062284]
table {%
80182 -6
80182 6
};
\addplot [very thick, red, opacity=0.04190039238878532]
table {%
80183 -6
80183 6
};
\addplot [very thick, red, opacity=0.06689497534160434]
table {%
80184 -6
80184 6
};
\addplot [very thick, red, opacity=0.00589885007072359]
table {%
80185 -6
80185 6
};
\addplot [very thick, red, opacity=0.12382897769665999]
table {%
80186 -6
80186 6
};
\addplot [very thick, red, opacity=0.0]
table {%
80187 -6
80187 6
};
\addplot [very thick, red, opacity=0.0]
table {%
80188 -6
80188 6
};
\addplot [very thick, red, opacity=0.0]
table {%
80189 -6
80189 6
};
\addplot [very thick, red, opacity=0.058633420674325955]
table {%
80190 -6
80190 6
};
\addplot [very thick, red, opacity=0.0065021116884940305]
table {%
80191 -6
80191 6
};
\addplot [very thick, red, opacity=0.10031860295489768]
table {%
80192 -6
80192 6
};
\addplot [very thick, red, opacity=0.009626899029060245]
table {%
80193 -6
80193 6
};
\addplot [very thick, red, opacity=0.003542072583900545]
table {%
80194 -6
80194 6
};
\addplot [very thick, red, opacity=0.14230375552922436]
table {%
80195 -6
80195 6
};
\addplot [very thick, red, opacity=0.0]
table {%
80196 -6
80196 6
};
\addplot [very thick, red, opacity=0.0]
table {%
80197 -6
80197 6
};
\addplot [very thick, red, opacity=0.0]
table {%
80198 -6
80198 6
};
\addplot [very thick, red, opacity=0.0]
table {%
80199 -6
80199 6
};
\addplot [very thick, red, opacity=0.0036683321046236724]
table {%
80200 -6
80200 6
};
\addplot [very thick, red, opacity=0.024950775444919613]
table {%
80201 -6
80201 6
};
\addplot [very thick, red, opacity=0.0059622847656091655]
table {%
80202 -6
80202 6
};
\addplot [very thick, red, opacity=0.02637606592621759]
table {%
80203 -6
80203 6
};
\addplot [very thick, red, opacity=0.0]
table {%
80204 -6
80204 6
};
\addplot [very thick, red, opacity=0.0]
table {%
80205 -6
80205 6
};
\addplot [very thick, red, opacity=0.029837326583917727]
table {%
80206 -6
80206 6
};
\addplot [very thick, red, opacity=0.0]
table {%
80207 -6
80207 6
};
\addplot [very thick, red, opacity=0.037401397239355726]
table {%
80208 -6
80208 6
};
\addplot [very thick, red, opacity=0.0]
table {%
80209 -6
80209 6
};
\addplot [very thick, red, opacity=0.019241733339421266]
table {%
80210 -6
80210 6
};
\addplot [very thick, red, opacity=0.004276426592504183]
table {%
80211 -6
80211 6
};
\addplot [very thick, red, opacity=0.024011898696766086]
table {%
80212 -6
80212 6
};
\addplot [very thick, red, opacity=0.004728738766159166]
table {%
80213 -6
80213 6
};
\addplot [very thick, red, opacity=0.0006066791864090867]
table {%
80214 -6
80214 6
};
\addplot [very thick, red, opacity=0.0]
table {%
80215 -6
80215 6
};
\addplot [very thick, red, opacity=0.0]
table {%
80216 -6
80216 6
};
\addplot [very thick, red, opacity=0.0007537005099458487]
table {%
80217 -6
80217 6
};
\addplot [very thick, red, opacity=0.0011765431700967164]
table {%
80218 -6
80218 6
};
\addplot [very thick, red, opacity=0.0]
table {%
80219 -6
80219 6
};
\addplot [very thick, red, opacity=0.0]
table {%
80220 -6
80220 6
};
\addplot [very thick, red, opacity=0.0]
table {%
80221 -6
80221 6
};
\addplot [very thick, red, opacity=0.0]
table {%
80222 -6
80222 6
};
\addplot [very thick, red, opacity=0.0]
table {%
80223 -6
80223 6
};
\addplot [very thick, red, opacity=0.015445374879873226]
table {%
80224 -6
80224 6
};
\addplot [very thick, red, opacity=0.0]
table {%
80225 -6
80225 6
};
\addplot [very thick, red, opacity=0.0]
table {%
80226 -6
80226 6
};
\addplot [very thick, red, opacity=0.0]
table {%
80227 -6
80227 6
};
\addplot [very thick, red, opacity=0.0]
table {%
80228 -6
80228 6
};
\addplot [very thick, red, opacity=0.00047121873668174585]
table {%
80229 -6
80229 6
};
\addplot [very thick, red, opacity=0.0]
table {%
80230 -6
80230 6
};
\addplot [very thick, red, opacity=0.0]
table {%
80231 -6
80231 6
};
\addplot [very thick, blue, opacity=0.005674155557083069]
table {%
80136 -6
80136 6
};
\addplot [very thick, blue, opacity=0.00308691509567359]
table {%
80137 -6
80137 6
};
\addplot [very thick, blue, opacity=0.0009760599412348207]
table {%
80138 -6
80138 6
};
\addplot [very thick, blue, opacity=0.0024537367065368566]
table {%
80139 -6
80139 6
};
\addplot [very thick, blue, opacity=0.0]
table {%
80140 -6
80140 6
};
\addplot [very thick, blue, opacity=0.014097852854334002]
table {%
80141 -6
80141 6
};
\addplot [very thick, blue, opacity=0.0]
table {%
80142 -6
80142 6
};
\addplot [very thick, blue, opacity=0.015509763982063603]
table {%
80143 -6
80143 6
};
\addplot [very thick, blue, opacity=0.0]
table {%
80144 -6
80144 6
};
\addplot [very thick, blue, opacity=0.002039636979081643]
table {%
80145 -6
80145 6
};
\addplot [very thick, blue, opacity=0.0]
table {%
80146 -6
80146 6
};
\addplot [very thick, blue, opacity=0.0017125189794594067]
table {%
80147 -6
80147 6
};
\addplot [very thick, blue, opacity=0.016690208289411486]
table {%
80148 -6
80148 6
};
\addplot [very thick, blue, opacity=3.033758651694479e-05]
table {%
80149 -6
80149 6
};
\addplot [very thick, blue, opacity=0.0]
table {%
80150 -6
80150 6
};
\addplot [very thick, blue, opacity=0.00024818665076182074]
table {%
80151 -6
80151 6
};
\addplot [very thick, blue, opacity=0.0]
table {%
80152 -6
80152 6
};
\addplot [very thick, blue, opacity=0.0011190660857667467]
table {%
80153 -6
80153 6
};
\addplot [very thick, blue, opacity=0.005359184288009813]
table {%
80154 -6
80154 6
};
\addplot [very thick, blue, opacity=0.07285952486953491]
table {%
80155 -6
80155 6
};
\addplot [very thick, blue, opacity=0.0016742919886734698]
table {%
80156 -6
80156 6
};
\addplot [very thick, blue, opacity=0.0]
table {%
80157 -6
80157 6
};
\addplot [very thick, blue, opacity=0.0009590336563570211]
table {%
80158 -6
80158 6
};
\addplot [very thick, blue, opacity=0.0009021525555477649]
table {%
80159 -6
80159 6
};
\addplot [very thick, blue, opacity=0.0]
table {%
80160 -6
80160 6
};
\addplot [very thick, blue, opacity=0.031450465309205276]
table {%
80161 -6
80161 6
};
\addplot [very thick, blue, opacity=0.0]
table {%
80162 -6
80162 6
};
\addplot [very thick, blue, opacity=0.02309279483043702]
table {%
80163 -6
80163 6
};
\addplot [very thick, blue, opacity=0.0]
table {%
80164 -6
80164 6
};
\addplot [very thick, blue, opacity=0.0013569863480400765]
table {%
80165 -6
80165 6
};
\addplot [very thick, blue, opacity=0.0]
table {%
80166 -6
80166 6
};
\addplot [very thick, blue, opacity=0.0013339168906403613]
table {%
80167 -6
80167 6
};
\addplot [very thick, blue, opacity=0.0]
table {%
80168 -6
80168 6
};
\addplot [very thick, blue, opacity=0.0001346044334048185]
table {%
80169 -6
80169 6
};
\addplot [very thick, blue, opacity=0.010534310173622186]
table {%
80170 -6
80170 6
};
\addplot [very thick, blue, opacity=0.0]
table {%
80171 -6
80171 6
};
\addplot [very thick, blue, opacity=0.0]
table {%
80172 -6
80172 6
};
\addplot [very thick, blue, opacity=0.00017671755703893593]
table {%
80173 -6
80173 6
};
\addplot [very thick, blue, opacity=0.0031577826995682875]
table {%
80174 -6
80174 6
};
\addplot [very thick, blue, opacity=0.06845704927932303]
table {%
80175 -6
80175 6
};
\addplot [very thick, blue, opacity=0.00459111495359738]
table {%
80176 -6
80176 6
};
\addplot [very thick, blue, opacity=0.026991610510234753]
table {%
80177 -6
80177 6
};
\addplot [very thick, blue, opacity=0.0024140818703632786]
table {%
80178 -6
80178 6
};
\addplot [very thick, blue, opacity=0.005936526206274132]
table {%
80179 -6
80179 6
};
\addplot [very thick, blue, opacity=0.0]
table {%
80180 -6
80180 6
};
\addplot [very thick, blue, opacity=0.0]
table {%
80181 -6
80181 6
};
\addplot [very thick, blue, opacity=0.0]
table {%
80182 -6
80182 6
};
\addplot [very thick, blue, opacity=0.0]
table {%
80183 -6
80183 6
};
\addplot [very thick, blue, opacity=0.0]
table {%
80184 -6
80184 6
};
\addplot [very thick, blue, opacity=0.0]
table {%
80185 -6
80185 6
};
\addplot [very thick, blue, opacity=0.0]
table {%
80186 -6
80186 6
};
\addplot [very thick, blue, opacity=0.0008848870862402341]
table {%
80187 -6
80187 6
};
\addplot [very thick, blue, opacity=0.0590274736433662]
table {%
80188 -6
80188 6
};
\addplot [very thick, blue, opacity=0.000594787147373093]
table {%
80189 -6
80189 6
};
\addplot [very thick, blue, opacity=0.0]
table {%
80190 -6
80190 6
};
\addplot [very thick, blue, opacity=0.0]
table {%
80191 -6
80191 6
};
\addplot [very thick, blue, opacity=0.0]
table {%
80192 -6
80192 6
};
\addplot [very thick, blue, opacity=0.0]
table {%
80193 -6
80193 6
};
\addplot [very thick, blue, opacity=0.0]
table {%
80194 -6
80194 6
};
\addplot [very thick, blue, opacity=0.0]
table {%
80195 -6
80195 6
};
\addplot [very thick, blue, opacity=0.008739663761985973]
table {%
80196 -6
80196 6
};
\addplot [very thick, blue, opacity=0.07219001446795013]
table {%
80197 -6
80197 6
};
\addplot [very thick, blue, opacity=0.005844868400577551]
table {%
80198 -6
80198 6
};
\addplot [very thick, blue, opacity=0.006531119527557958]
table {%
80199 -6
80199 6
};
\addplot [very thick, blue, opacity=0.0]
table {%
80200 -6
80200 6
};
\addplot [very thick, blue, opacity=0.0]
table {%
80201 -6
80201 6
};
\addplot [very thick, blue, opacity=0.0]
table {%
80202 -6
80202 6
};
\addplot [very thick, blue, opacity=0.0]
table {%
80203 -6
80203 6
};
\addplot [very thick, blue, opacity=0.07486122836453553]
table {%
80204 -6
80204 6
};
\addplot [very thick, blue, opacity=0.004841703637689665]
table {%
80205 -6
80205 6
};
\addplot [very thick, blue, opacity=0.0]
table {%
80206 -6
80206 6
};
\addplot [very thick, blue, opacity=0.0034392922611195813]
table {%
80207 -6
80207 6
};
\addplot [very thick, blue, opacity=0.0]
table {%
80208 -6
80208 6
};
\addplot [very thick, blue, opacity=0.00048491105140513165]
table {%
80209 -6
80209 6
};
\addplot [very thick, blue, opacity=0.0]
table {%
80210 -6
80210 6
};
\addplot [very thick, blue, opacity=0.0]
table {%
80211 -6
80211 6
};
\addplot [very thick, blue, opacity=0.0]
table {%
80212 -6
80212 6
};
\addplot [very thick, blue, opacity=0.0]
table {%
80213 -6
80213 6
};
\addplot [very thick, blue, opacity=0.0]
table {%
80214 -6
80214 6
};
\addplot [very thick, blue, opacity=0.08978046294675848]
table {%
80215 -6
80215 6
};
\addplot [very thick, blue, opacity=0.00020037255849485086]
table {%
80216 -6
80216 6
};
\addplot [very thick, blue, opacity=0.0]
table {%
80217 -6
80217 6
};
\addplot [very thick, blue, opacity=0.0]
table {%
80218 -6
80218 6
};
\addplot [very thick, blue, opacity=0.004660156586969096]
table {%
80219 -6
80219 6
};
\addplot [very thick, blue, opacity=0.0027885847128431658]
table {%
80220 -6
80220 6
};
\addplot [very thick, blue, opacity=0.06904794420097883]
table {%
80221 -6
80221 6
};
\addplot [very thick, blue, opacity=0.01104289847000495]
table {%
80222 -6
80222 6
};
\addplot [very thick, blue, opacity=0.06580841408210215]
table {%
80223 -6
80223 6
};
\addplot [very thick, blue, opacity=0.0]
table {%
80224 -6
80224 6
};
\addplot [very thick, blue, opacity=0.3352361532485421]
table {%
80225 -6
80225 6
};
\addplot [very thick, blue, opacity=0.220079843644861]
table {%
80226 -6
80226 6
};
\addplot [very thick, blue, opacity=0.002209292372962676]
table {%
80227 -6
80227 6
};
\addplot [very thick, blue, opacity=0.0800818538904158]
table {%
80228 -6
80228 6
};
\addplot [very thick, blue, opacity=0.0]
table {%
80229 -6
80229 6
};
\addplot [very thick, blue, opacity=0.0023160033450424756]
table {%
80230 -6
80230 6
};
\addplot [very thick, blue, opacity=0.045710840074253266]
table {%
80231 -6
80231 6
};
\end{groupplot}

\draw ({$(current bounding box.south west)!0.5!(current bounding box.south east)$}|-{$(current bounding box.south west)!0.98!(current bounding box.north west)$}) node[
  scale=0.6,
  anchor=north,
  text=black,
  rotate=0.0
]{TIG Attributions for -loc, predictions with NHITS};
\end{tikzpicture}
